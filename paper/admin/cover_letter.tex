\title{cover letter}
%
% See http://texblog.org/2013/11/11/latexs-alternative-letter-class-newlfm/
% and http://www.ctan.org/tex-archive/macros/latex/contrib/newlfm
% for more information.
%
\documentclass[10pt,stdletter,orderfromtodate,sigleft]{newlfm}
\usepackage{pxfonts, geometry}

  \setlength{\voffset}{0in}

\newlfmP{dateskipbefore=0pt}
\newlfmP{sigsize=20pt}
\newlfmP{sigskipbefore=10pt}
 
\newlfmP{Headlinewd=0pt,Footlinewd=0pt}

\newcommand{\journal}{Science Advances}
\newcommand{\articletype}{Research Article}
\newcommand{\myTitle}{High-level cognition is supported by information-rich but compressible brain activity patterns}
\newcommand{\corresponding}{Jeremy R. Manning}


\namefrom{\vspace{-0.3in}\corresponding}
\addrfrom{
  Dartmouth College\\
  Department of Psychological \& Brain Sciences\\
  Moore Hall\\
  Hanover, NH, 03755}
\addrto{}
\dateset{\today}

 
\greetto{To the editors of \textit{\journal}:}


 
\closeline{Sincerely,}
%\usepackage{setspace}
%\linespread{0.85}
% How will your work make others in the field think differently and move the field forward?
% How does your work relate to the current literature on the topic?
% Who do you consider to be the most relevant audience for this work?
% Have you made clear in the letter what the work has and has not achieved?

\begin{document}
\begin{newlfm}

  We have enclosed our manuscript entitled \textit{\myTitle} to be considered
  for publication as an \textit{\articletype}. Our manuscript uses
  dimensionality reduction algorithms and pattern classifiers to explore how
  informative and compressible (in the information theory senses of those
  words) brain activity patterns are, under different cognitive circumstances.

  We developed a computational framework for evaluating our two measures of
  interest (informativeness and compressibility), and we applied the framework
  to a public fMRI dataset. In the dataset's experiment, participants either
  listened to an auditory recording of a story, listened to temporally
  scrambled recordings of the story (randomizing the orders of paragraphs or
  words), or underwent a resting state scan. Each condition is designed to
  engage cognitive processing and engagement at different depths. For example,
  listening to the intact story leads participants to mentally engage with the
  narrative, and to deeply process and connect narrative events in a way that
  leads to a rich understanding of the story. Listening to temporally scrambled
  versions of the story might lead to understanding individual paragraphs or
  words. However, individual moments in the temporally scrambled stories lack
  contextual elements that enable participants to gain a deeper understanding.
  Across conditions, we found that both informativeness and compressibility of
  the brain patterns changed systematically with the cognitive ``richness'' of
  the stimulus across the different experimental conditions. We also traced out
  the brain networks associated with these changes. We found that networks
  traditionally associated with higher-level cognitive functions tended to
  exhibit more information-rich brain patterns than networks traditionally
  associated with lower-level cognitive functions.

  Taken together, our work provides new insights into the fundamental ``rules''
  describing how our brains respond to and represent specific stimuli and
  cognitive processes. We connect our findings about how the informativeness
  and compressibility of brain activity change across experimental conditions
  with prior work on task-dependent changes in functional connectivity (i.e.,
  full-brain correlations). Our work helps to clarify how the ``neural code''
  might be structured, and how the neural code might vary across tasks and
  brain areas. We expect that our work will be of interest to a broad audience
  including neuroscientists, cognitive psychologists and cognitive scientists,
  and others.


Thank you for considering our manuscript, and I hope you will find it suitable
for publication in \textit{\journal}.


\end{newlfm}
\end{document}
